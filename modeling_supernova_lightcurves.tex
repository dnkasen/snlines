\documentclass{tufte-handout}

\input{header}
\title{Calculating Analytic Light Curves of Supernovae, Kilonovae, and Other Transients: A worksheet}
\author{D. Kasen (kasen@berkeley.edu)}
%\author[The Tufte-LaTeX Developers]{The Tufte-\LaTeX\ Developers}
\date{}
%\date{due Friday, 1/27/2017, 5 PM}   % if the \date{} command is left out, the current date will be used

\newcommand{\texp}{\ensuremath{t_{\rm exp}}}
\newcommand{\rph}{\ensuremath{R_{\rm ph}}}
\newcommand{\Eint}{\ensuremath{E_{\rm int}}}
\newcommand{\tlc}{\ensuremath{t_{\rm lc}}}

\begin{document}

\maketitle

\newcounter{subq}
\newcounter{quest}
\setcounter{quest}{0}


\section{Introduction and Approximations}

Here we develop a simple, one-zone (semi-)analytic model for the light curves
from the expanding debris of a supernova, kilonova or related sort of explosive transient.
After completing the worksheet, one should (hopefully) be able to understand and calculate models that can be applied to theoretical and observational research studies.

 Our discussion
follows the analytic methods
for supernova light curves presented in the  papers of
Arnett (1980, 1982).
Consider a cloud of gas with mass $M$ ejected from some explosion with
a  kinetic energy $E_K$.  The fundamental equation
describing the evolution of the expanding cloud of ejecta is the first law of
thermodynamics, which expresses energy conservation

\beq
\od{\Eint(t)}{t} = - p \od{V}{t} + \dot{Q}(t) - L(t)
\label{eq:firstlaw}
\eeq
where \Eint\ is the total internal energy of the ejecta of volume $V$. The $p dV/dt$
term is the rate of work done
by the ejecta pressure $p$, $\dot{Q}$ is the total heating input rate (erg~s$^{-1}$) due to radioactivity (or any other energy input sources) and $L$ is the  
luminosity (erg~s$^{-1}$) escaping the system (which is light curve we would like to calculate).


We can solve Eq.~\ref{eq:firstlaw} and so calculate the light curve $L(t)$ provide we make some simplifying assumptions.

\pp {\bf Assumption 1:} We consider a spherical, one-zone model of the ejecta, i.e.,
a spherical uniform density cloud of radius $R$. This is obviously a gross approximation
of the actual remnant structure, but still turns out to give
insightful scaling relations.

\pp {\bf Assumption 2:} The ejecta  expands freely as $R = v t + R_0$ where
$v = \sqrt{2E_K/M}$ is the characteristic velocity, $R_0$ is the radius of the system
at the time of ejection\footnote{For example, $R_0$ is the radius of the progenitor star. In most cases  we are interested at times when $v t \gg R_0$ and can simply
take $R = vt$. Still, we retain  $R_0$ for now for completeness.}, and $t$ is the time since
ejection.  The volume of the cloud is 
\beq
V(t)  = \frac{4 \pi}{3} (vt + R_0)^3
\eeq
and the density is $\rho(t) = M/V$.

\pp {\bf Assumption 3:} Radiation energy dominates over gas energy. The internal energy density, $u$, and pressure are then
\beq
u  = aT^4~~~~~~~~~p = \frac{1}{3} a T^4
\eeq
In our one zone model, $u = \Eint/V$. 
We will be able to confirm that the assumption of radiation domination is well
motivated given the temperatures and densities we  find for the
ejecta.


\pp {\bf Assumption 4:}
The radiation leaking from the cloud can be described by the
standard diffusion equation in spherical coordinates\footnote{Diffusion only applies at
 epochs when the ejecta is still reasonably optically thick.}
\beq
L(r) = -4 \pi r^2 \frac{c}{3 \kappa \rho} \pd {u}{r}
\label{eq:diff}
\eeq
where $\kappa$ is the opacity. 
To properly calculate the derivative of $u = aT^4$ we would need to know the radial run of
 temperature. However, since we are using a
one-zone model we can approximate\footnote{This simple approximation is equivalent to assuming $u$ declines as
\beq
u(r) = \frac{\Eint}{V} \left(1 - \frac{r}{R} \right)
\eeq
i.e., the energy density drops linearly from $\Eint/V$ at the center to $0$ at the outer edge.
} 
the spatial derivative as 
\beq
\pd {u}{r} \sim -\frac{\Eint/V}{R}
\eeq


\pp {\bf Assumption 5: }The opacity, $\kappa$, is a constant. Of course, in reality the opacity is a complicated function of time, radius, and wavelength; nevertheless, we can try to identify an
average effective grey opacity.  For ordinary supernova material, a reasonable value is
$\kappa = 0.1~{\rm cm^2~g^{-1}}$. 
For the much more opaque heavy r-process ejecta from
neutron star mergers, the value is more like $\kappa = 10~{\rm cm^2~g^{-1}}$. 

\bigskip
\pp {\it {\LARGE Problems:}}
\bigskip



\pp A supernova light curve represents thermal radiation diffusing out of an expanding, hot cloud of ejecta. The diffusion time\footnote{In an optically thick medium, a photon
steps (on average) one mean free path
$l = 1/\kappa \rho$ before scattering. In a random walk diffusion process, after  $N$ steps the photon has only moved an average distance $R = \sqrt{N} l$. Thus, $N = R^2/l^2$ steps are 
needed for the photon to escape a region of radius $R$, and the diffusion time is 
\beq
\td = N \frac{l}{c} = \frac{R^2}{l c} = \frac{ R^2 \kappa \rho}{c} = \tau \frac{R}{c}.
\eeq
So the diffusion time is simply a factor of $\tau = R \kappa \rho$ longer than the free-streaming timescale $R/c$.
}
through a static medium of radius of $R$ is  $\td = \tau R/c$
where the optical depth $\tau = \kappa \rho R$.  In a moving medium, photon escape is modified by the fact that cloud radius is growing and the density declining over time. The timescale at 
which photons
effectively escape is when the diffusion time, $ \td$, becomes comparable to the current expansion
time $t_{\rm exp} = R/v$ of the  cloud.  

\sq{quest} Setting $\td \approx t_{\rm exp}$, show that  the timescale, \tlc, at which 
photons escape is
\beq
\tlc =
\left[ \frac{3}{4 \pi} \frac{M \kappa} {v c} \right]^{1/2}
\propto M^{3/4} \kappa^{1/2} E_k^{-1/4}
\eeq
where we assumed $vt \gg R_0$. 


\comment The timescale \tlc\ provides an estimate of the peak time of the transient 
light curve. While the scaling with physical parameters is reasonable, the numerical constant out front would need to be calibrated to achieve some quantitative accuracy. 

\sq{quest} What is the optical depth of the ejecta at the time \tlc\ when
photons can effectively escape?

\comment You see that photons begin to effectively escape for  $\tau > 1$. The condition for escape is not that the ejecta be optically thin, but that photons can diffuse out faster than the ejecta itself grows.

\sq{quest} Show that the luminosity from the diffusion equation (Eq.~\ref{eq:diff}) can 
be written
\beq
L(t) \approx \frac{\Eint(t) }{\tlc^2}  (t + t_0)
\eeq
where $t_0 = R_0/v$ is the timescale\footnote{As mentioned, we will often be interested in times $t \gg t_0$, or equivalently $v t \gg R_0$, in which case we
can drop the $t_0$ in this expression.} it took the ejecta to expand to double its initial size $R_0$. To calculate the light curve  we just need to solve for how 
$\Eint(t)$ evolves with time. Let's do it step by step.

\sq{quest}  Consider first the evolution of \Eint\ in the case of adiabatic expansion where 
$L = \dot{Q} = 0$
(i.e., no heat is entering or leaving the system). Show that
\beq
\Eint(t) = E_0 \left( \frac{t_0}{t + t_0}  \right) 
\eeq
where $E_0$ is the initial internal energy and time $t = 0$ (corresponding to the end of the explosion and the start of the ejecta expansion).  Thus, the internal
energy cools adiabaticaly like
$\Eint \propto R_0/R \propto 1/t$ .

\comment In a stellar explosion powered by a strong shock, the initial internal energy $E_0$ is
typically comparable to the kinetic energy $E_K$.  However, over time the internal energy turns
 into kinetic energy as  pressure does $p dV$ work on the ejecta.  At times $t \gg t_0$, the internal energy has dropped to but a small fraction\footnote{Unless some source heats up the ejecta substantially.} of the kinetic energy, $\Eint \ll E_K$, and there is no longer enough internal energy to significantly accelerate the ejecta\footnote{Or equivalently, the ejecta becomes highly supersonic as the sound speed $c_s$ in the cooled ejecta becomes $c_s \ll v$. Pressure waves can longer move effectively move around to accelerate the ejecta.}. Thus, free homologous expansion becomes an increasingly excellent assumption
 at times $t \gg t_0$.





%\sq{quest} A rough estimate of the peak luminosity of a light curve without heating
%$(\dot{q} = 0$) is  $L_{\rm lc} \approx \Eint(\tlc)/\tlc$. Write an expression for $L_{\rm lc}$ in
%terms of $E_0, \kappa, M, R_0, v$.




\sq{quest} Next, consider the case where there is no heating ($\dot{Q} = 0$) but 
radiation can escape the remnant ($L \ne 0$). 
Solve equation~\ref{eq:firstlaw} for 
$\Eint(t)$ and derive the analytic formula for the light curve
\beq
L(t) = \frac{ E_0}{\tlc} \frac{R_0}{v \tlc}  \exp\left[ - \frac{(t + t_0)^2}{2 \tlc^2} \right]~~~~{\rm (no~heating)}
\label{eq:noheat}
\eeq

\comment As expected, the quantity $\tlc$ gives the characteristic duration of the lightcurve of Eq.~\ref{eq:noheat}. The coefficient in front of the exponential is an estimate of the characteristic light curve
luminosity. It is simply the total initial energy $E_0$ divided by the timescale for this  energy to leak out, \tlc,
but {\it modified} by a factor $R_0/v \tlc$ that accounts for the loss of internal energy due to adiabatic expansion.

\sq{quest} Write the coefficient in front of Eq.~\ref{eq:noheat} in terms of $R_0, M$ and $E_K$ to see how the (no-heating) luminosity scales with physical parameters. Plug in default values  $M = M_\odot, R = R_\odot, E_0 = 10^{51}$~erg, and scale to discuss what luminosity is expected for a Type~IIP supernova (massive red giant explosion), a Type~Ia supernova (white dwarf) and a kilonova (neutron star merger ejecta). You'll see that a light curve powered by the diffusion of energy $E_0$ deposited in the explosion is very dim except for stars with large $R_0$. In other cases, continuing energy input is needed.

\sq{quest} Finally, consider the general case where there is both radioactive or other 
heating $(\dot{Q} \ne 0)$ and radiation $(L \ne 0)$. Rearrange the equations to show that
\beq
\od{L(t)}{t} = \frac{(t + t_0)}{\tlc^2} [ \dot{Q(t)} - L(t)] 
\label{eq:Larn}
\eeq

\comment Notice that according to Eq.~\ref{eq:Larn}, at the maximum of the light curve  (where $dL/dt = 0$)  we must
have $L = \dot{Q}$. In other words, the luminosity at the peak of the light curve is 
equal to the instantaneous rate of heating at that time. You have just proven what
is know as "Arnett's Law". It is only roughly true given the approximations we have made,
but a useful rule of thumb.

\sq{quest}  There is no simple, general solution for Eq.~\ref{eq:Larn}, but you
can write the light curve as an integral over $\dot{Q}(t)$.  Consider
attacking the differential equation using an \link{https://en.wikipedia.org/wiki/Integrating_factor}{integrating factor} and show that\footnote{We now assume $t \gg t_0$, and so drop $t_0$, but it is straightforward
to include it if you wish.}
\beq
L(t) = \exp[-\frac{t^2}{2 \tlc^2}]
\left( \frac{E_0 R_0}{v \tlc^2}  + 
 \int_0^t 
\dot{Q}(t') 
\left( \frac{t'}{\tlc^2}\right)
\exp[\frac{t'^2}{2 \tlc^2}]
dt' \right)
\eeq
Which uses the initial condition for $L(0)$ given by Eq.~\ref{eq:noheat}.
In the limit of no heating ($\dot{Q}(t) = 0$) we get only the first term in parenthesis, which reduces to the no-heating result
Eq.~\ref{eq:noheat}. In the other limit that the initial energy contribution $E_0 t_0/\tlc^2$ is negligible the above simplifies to
\beq
L(t) = 
e^{-(t^2/2\tlc^2)}
 \int_0^t 
 \frac{\dot{Q}(t') t'}{\tlc^2}
e^{(t'^2/2\tlc^2)}
dt'~~~~{\rm (with~heating)}
\label{eq:soln}
\eeq
This integral is relatively easy to calculate numerically, given some heating function $\dot{Q}(t)$.

\sq{quest}  Consider the earliest epochs of the light curve, when $t \ll \tlc$ and the heating
rate can be assumed to be roughly constant $\dot{Q} = \dot{Q}(0)$. Expand  the exponentials in Eq.~\ref{eq:soln} to
show that
\beq
L(t \ll \tlc) \approx Q_0 \left( \frac{t}{\tlc} \right)^2  + \mathcal{O}(t/\tlc)^4 
\eeq
So to the leading term we have a light curve rise of $L(t) \propto t^2$ at early times. You could keep higher order terms if you care to.

\comment Sometimes the scaling $L \propto t^2$
is explained by appealing to a "fireball" model, where at early times the photospheric radius grows as $R_{\rm ph} = v_{\rm ph} t$ while the photospheric temperature $T_{\rm ph}$ remains constant -- hence the luminosity is $L = 4 \pi R_{\rm ph}^2 \sigma_{\rm sb} T_{\rm ph}^4$ scales as $t^2$. The above derivation shows that the $t^2$ scaling can be derived from more physical
considerations. 

The $L \propto t^2$ behavior often gives a reasonable match to the early rise of observed supernova light curves, perhaps fortuitously given the many approximations we have made. More meaningfuls estimates of the light curve rise require specifying the radial profile of density and heating input.

\sq{quest} Write a simple code to integrate Eq.~\ref{eq:soln} and so calculate a light curve. For example, the heating due to radioactive $^{56}$Ni heating is  
\beq
\dot{q}_{\rm ni}(t) = 3.9 \times 10^{10} e^{-t/\tau_{\rm ni}} + 6.78 \times 10^9 \left[ e^{-t/t_{\rm co}} 
-e^{-t/t_{\rm ni}} \right]~~{\rm erg~g~s^{-1}}
\eeq
where $t_{\rm ni} = 8.8$~days and $t_{\rm co} = 113.6$~days. The heating for radioactive r-process decay is
\beq
\dot{q}_{\rm rp}(t) \approx 2 \times 10^{10} \left( \frac{t}{1~{\rm day}} \right)^{-1.3}~{\rm erg~g~s^{-1}}
\eeq
The heating from central magnetar spindown is
\beq
\dot{q}_{\rm mag}(t) = \frac{E_m/t_m}{(1 + t/t_m)^2}{\rm erg~s^{-1}}
\eeq
where $E_m$ is the total rotational energy of the magnetar and $t_m$ the spindown timescale.


\sq{quest} The above model only allows us to calculate a bolometric light curve. In order to estimate the colors, we can make the (very rough) approximation that the spectrum is a blackbody\footnote{The opacity in supernova ejecta is typical strongly wavelength dependent, so the photospheric radius is a function of wavelength, hence blackbody emission
at a single temperature is not a great assumption. But at least it is something we can do easily.} at the  temperature at the photosphere, $T_{\rm ph}$, which can be determined by
\beq
T_{\rm ph}(t) = \left[ \frac{L}{4 \pi  R_{\rm ph} \sigma_{\rm sb}} \right]^{1/4}
\eeq
To derive the photospheric radius, $R_{\rm ph}(t)$, we must assume something about
the density structure of the ejecta. In the absence of a detailed explosion model, we can
use simple analytic profiles, two of which are given below.

\addtocounter{subq}{1}
%
%In homology, the density structure of the ejecta retains a self-similar form with time, 
%with the overall level declining as $\rho \propto t^{-3}$.  The shape of the density
%profile is predicted by detailed a hydrodynamical explosion models,  but
%it is convenient to use simple analytic forms which have been shown to
% reasonably approximate some of the detailed 
% calculations. The analytic forms allow us to describe the ejecta
%using just a few parameters: the total ejecta mass, $M$, and the ejecta kinetic energy $E$. 

\bigskip \noindent {\bf Exponential Density Profile}: Consider the density profile

\beq
\rho(r,t) = \rho_0(t) e^{-v /v_e}
\eeq
where $\rho_0(t)$ and $v_e$ are parameters and we assume $r = vt$. In homologous expansion, the shape of the density profile does not change with time, it simple scales outward and drops as $\rho \propto t^{-3}$.

\Q{quest}{subq} Integrate the density profile over radius\footnote{The total mass in an infinitesimal shell is $dm = 4 \pi r^2 \rho(r) dr$ and the total kinetic energy is $dE =  \frac{1}{2} 4 \pi r^2 \rho(r)  v^2 dr$} (at any given time) and determine $\rho_0(t)$ and $v_e$ in terms of
$M$, $E_k$, and $t$.

\pp The continuum optical depth integrated outward from some radius $r$ to infinity is
\beq
\tau(r) = \int_r^\infty \rho(r) \kappa dr
\eeq
where $\kappa$ is the continuum opacity. The photosphere is defined\footnote{Sometimes the photosphere is defined as $\tau = 2/3$ but the difference is not of great importance here.} as the radius where $\tau(r) = 1$. 

\Q{quest}{subq} Assuming a constant opacity $\kappa$, determine the photospheric radius and photospheric velocity in terms of $M, E_k$, and $t$. 

\Q{quest}{subq} Determine the time after which $\tau < 1$ throughout the entire ejecta. This marks the onset of the "nebular" phase, where the assumption of blackbody emission becomes severely bad.

\bigskip \noindent {\bf Broken Power Law Density Profile}: An alternative simple density structure is the broken power-law, given by

\begin{eqnarray}
\rho(r,t) = \rho_0(t) (v/v_t)^{-\delta}~~{\rm for~} v < v_t
 \\
\rho(r,t) = \rho_0(t) (v/v_t)^{-n}~~ {\rm for~} v \geq v_t
\end{eqnarray}
Supernova simulations show that the ejecta is often described by a shallow
exponent $\delta$ in the interior layers and a steep exponent $n$ in the outer layers.

\Q{quest}{subq} What are the allowed values of $n$ and $\delta$ such that the total mass and energy converge? 


\Q{quest}{subq} Integrate the density profile over radius to determine $\rho_0$ and $v_t$ in terms of
 $M,E, n$ and $\delta$. 

\Q{quest}{subq} Calculate the photospheric radius and velocity over time.



\pp In the blackbody model, the specific luminosity $L_\nu$ (units 
${\rm ergs~s^{-1}~Hz^{-1}}$) at some frequency $\nu$ is\footnote{It is common to forget
the factor of $\pi$ in this expression, but it necessary since 
\beq
\int_{0}^{\infty} B_\nu(T) d\nu = \frac{ \sigma_{\rm sb} T^4}{\pi}
\eeq
and so is needed to assure that $\int L_\nu d\nu = L$. }
\beq
L_\nu(\nu) =B_\nu(T_{\rm ph})  \frac{\pi L}{\sigma_{\rm sb} T_{\rm ph}^4}
\eeq

\Q{quest}{subq} Generalize your code for the bolometric light curve $L(t)$ so that it also  calculates the photospheric radius, photospheric temperature, and specific luminosity $L_\nu$ as a function of time.  

\comment If you want to convert your specific luminosity into an apparent AB-magnitude, use the
astronomer's relation
\beq
M_{\rm AB} = -2.5 \log_{10} \left[ \frac{L_{\nu}}{4 \pi D^2} \right] - 48.6
\eeq
where $D$ is the distance to the source in cm. If you want to calculate an absolute
AB~magnitude, the convention is to use $D = 10$~parsecs~$= 3.086 \times 10^{19}$~cm.

\comment The semi-analytic light curve code you have developed here is capable of 
predicted light curves for theoretical models, or for fitting observational data.
In fact, such models have been used in many, many research papers. However, before over-interpreting your results, keep in mind the limitations imposed by the several simplifying assumptions we have made.


\end{document}


